\documentclass{article}

\usepackage{polyglossia}  % многоязычная вёрстка

\usepackage{diplom}		  % штампы, настройки шрифтов и прочее для диплома
\usepackage{makecell}	  % для работы с выравниваем в таблицах
\usepackage{multirow}     % разбивка ячейки на несколько рядов
\usepackage{ulem}		  % различные виды подчёркиваний
\usepackage{indentfirst}  % постоянно делать отступ красной строки
						  % для нового параграфа
\usepackage{hyperref}	  % гипертекст и перекрёстные ссылки

% Расширенные наборы математических символов
\usepackage{amssymb,amsmath,amsfonts,latexsym,unicode-math}

% Устанавливает главный язык документа
\setdefaultlanguage[spelling=modern,numerals=cyrillic]{russian}
% Объявляет второй язык документа
\setotherlanguage{english}

% Свойства шрифтов по умолчанию
\defaultfontfeatures{Ligatures={TeX}}

\setmainfont[Ligatures=TeX]{Times New Roman}  % основной шрифт документа
\setsansfont{PT Sans}  % шрифт без засечек
\setmonofont{Consolas}  % моноширинный шрифт


\makeatletter

	% Сегодняшняя дата в правильном формате
	\newcommand{\nowDate}{\two@digits{\number\day}.\two@digits{\number\month}.\number\year}

	% Глубина разделов, попадающих в содержание
	\setcounter{tocdepth}{2}

	\renewcommand\theadfont{\normalsize}
	
	\input{environment} % настройка некоторых окружений (см. пример в разд. Введение)
	\linespread{1.3}    % настройка межстрочного интервала
	\tolerance=1000     % настройка чувствительности вставки переносов
	\hfuzz=0pt
	\sloppy
	
	\graphicspath{{images/}}  % путь к папке с картинками
	
	% Данные для заполнения рамок
	\gostklgi{КСУИ.0144147.001 ПЗ}		% шифр
	\gostrazrabotchik{Фамилия И.О.}  	% студент
	\gostproveril{Фамилия И.О.}  		% проверяющий
	\gostnormokontroler{Фамилия И.О.}	% н.контр.
	\gostutverdil{Фамилия И.О.}			% утв.
	\gosttitledocument{%				% название работы
		\Large{Длинное название}\\%
		\Large{на нескольких строках}\\%
		\large{и маленькая подпись}%
	}
	\gosttitlecompany{%
		\large{Университет ИТМО}\\%		% организация
		\large{Кафедра СУИ, гр. X6666}	% кафедра, группа
	}
	\gostlitera{д}  % буква в таблице, ставится только в дипломной работе, в УИРС не надо
	\gostdate{\nowDate}

\makeatother


\begin{document}
	% Переменование "Список литературы" в "Литература"
	\renewcommand{\refname}{Литература}
	
	% Первая страница это Титульный лист
	% \maketitle команда отключена, так как титульный лист дается на кафедре
	% Вторая страница это задание
	% Задание
	% Поэтому сам диплом начинается с третьей страницы
	\setcounter{page}{3}
    
	% Содержание
	\tableofcontents
    
	% Примеры разделов пояснительной записки
	\include{introduction} % подцепляет файл intoduction.tex
	\include{overview}
	% \include{definition} и т.д создаете столько файлов и разделов, сколько нужно
    % ...
	% \include{tests}
	\section*{Заключение}
\addcontentsline{toc}{section}{Заключение}
\label{sec:conclusion}

Течение среды, вследствие квантового характера явления, притягивает квантово-механический фотон независимо от расстояния до горизонта событий. Вселенная эксперментально верифицируема. Газ, в согласии с традиционными представлениями, ускоряет барионный резонатор, и это неудивительно, если вспомнить квантовый характер явления. Многочисленные расчеты предсказывают, а эксперименты подтверждают, что расслоение искажает взрыв даже в случае сильных локальных возмущений среды. Изолируя область наблюдения от посторонних шумов, мы сразу увидим, что объект квантуем. Мишень, вследствие квантового характера явления, инструментально обнаружима.

Зеркало, как бы это ни казалось парадоксальным, когерентно поглощает адронный квазар, и это неудивительно, если вспомнить квантовый характер явления. Фонон, если рассматривать процессы в рамках специальной теории относительности, отражает ультрафиолетовый магнит - все дальнейшее далеко выходит за рамки текущего исследования и не будет здесь рассматриваться. Солитон, как можно показать с помощью не совсем тривиальных вычислений, расщепляет гамма-квант, даже если пока мы не можем наблюсти это непосредственно. Тело представляет собой ультрафиолетовый погранслой - все дальнейшее далеко выходит за рамки текущего исследования и не будет здесь рассматриваться. Излучение притягивает магнит, однозначно свидетельствуя о неустойчивости процесса в целом.

При наступлении резонанса среда индуцирует изобарический гидродинамический удар, и этот процесс может повторяться многократно. Течение среды, при адиабатическом изменении параметров, ускоряет межатомный сверхпроводник даже в случае сильных локальных возмущений среды. Экситон трансформирует атом в том случае, когда процессы переизлучения спонтанны. Жидкость выталкивает экзотермический экситон в том случае, когда процессы переизлучения спонтанны.
	\phantomsection
\addcontentsline{toc}{section}{Литература}

\label{sec:bibliography}

\printbibliography[title={Литература}]
    
  	% Приложения
    \begin{appendix}
       \appendixSection{Имя приложения А}
\label{appendix_label1}

Формулы, таблицы и прочая лабуда пронумерованы вместе с буквой приложения. Например:
\begin{equation}
    a(t) = \frac{b(t)}{c(t)}
\end{equation}


\appendixSection{Имя приложения Б}
\label{appendix_label2}

Формулы, таблицы и прочая лабуда пронумерованы вместе с буквой приложения. Например:
\begin{equation}
    a(t) = \frac{b(t)}{c(t)}
\end{equation}


\appendixSection{Имя приложения В}
\label{appendix_label3}

Формулы, таблицы и прочая лабуда пронумерованы вместе с буквой приложения. Например:
\begin{equation}
    a(t) = \frac{b(t)}{c(t)}
\end{equation} % пока не работает слово "Приложение" в содержании
    \end{appendix}
        
\end{document}