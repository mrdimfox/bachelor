\usepackage{ifxetex}

\ifxetex
    \usepackage{polyglossia}                    % многоязычная вёрстка
	
% Устанавливает главный язык документа
\setdefaultlanguage[spelling=modern,numerals=cyrillic]{russian}
% Объявляет второй язык документа
\setotherlanguage{english}

\else
	\usepackage{cmap}                               % Улучшенный поиск русских слов в полученном pdf-файле
\defaulthyphenchar=127                          % Если стоит до fontenc, то переносы не впишутся в выделяемый текст при копировании его в буфер обмена 
\usepackage[T1,T2A]{fontenc}                    % Поддержка русских букв
\usepackage[utf8]{inputenc}                     % Кодировка utf8
\usepackage[english, russian]{babel}            % Языки: русский, английский

\usepackage{paratype}

\fi

\usepackage{makecell}	  % для работы с выравниваем в таблицах
\usepackage{multirow}     % разбивка ячейки на несколько рядов
\usepackage{indentfirst}  % постоянно делать отступ красной строки
						  % для нового параграфа
\usepackage{hyperref}	  % гипертекст и перекрёстные ссылки

% Расширенные наборы математических символов
\usepackage{amssymb,amsmath,amsfonts,latexsym}
\ifxetex
    \usepackage{unicode-math}
\fi

% Всё для библиографии
\usepackage[
    backend=biber,%
    style=gost-numeric,%
    language=auto,%
    sorting=none,%
    autolang=other
]{biblatex}							% основной пакет для работы с библиографией
\usepackage{csquotes}	  			% умные кавычки
\addbibresource{bibliography.bib}	% файл с библиографией

\usepackage{styles/diplom}		  % штампы, настройки шрифтов и прочее для диплома

\ifxetex
    % Свойства шрифтов по умолчанию
\defaultfontfeatures{Ligatures={TeX}}

\setmainfont[Ligatures=TeX]{Times New Roman}  % основной шрифт документа
\setsansfont{PT Sans}                         % шрифт без засечек
\setmonofont{Consolas}                        % моноширинный шрифт
\fi